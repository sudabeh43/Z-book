\chapter{چند یادآوری اساسی}\label{appendix1}
\paragraphfootnotes
جسم‌های حاصل از دوران، جسم‌هایی هستند که شکل آن‌ها از دوران حول محور‌ها به دست می‌آید. گاهی جسم‌های تولید شده، جسم‌هایی هستند که با استفاده از فرمول‌های هندسه، به راحتی می‌توانیم حجم آن‌ها را حساب کنیم؛ اما گاهی شکل این جسم‌ها، منظم نیست و لذا ناچاریم برای محاسبهٔ حجم آن‌ها از حساب دیفرانسیل و انتگرال کمک بگیریم. ناحیه ممکن است  دارای شکل خاصی باشد که در این صورت، با استفاده از فرمول‌های  هندسه می‌توانیم مساحت\index{مساحت} آن‌را حساب کنیم.
\section{استقرای ریاضی و چند مثال}
وقتی ناحیه‌ای توسط منحنی‌هایی که یکدیگر را قطع می‌کنند، مشخص می‌شود، 
نقاط تقاطع\index{نقاط تقاطع}، حدود انتگرال‌گیری را تعیین می‌کنند. مثال بعدی، نمونه‌ای از این 
حالت را نشان می‌دهد. سوالی که ممکن است در اینجا پیش بیاید این است که کدام‌یک از ۳ روش گفته شده، بهتر است؟ واقعیت این است که به طور قطع، نمی‌توان گفت که کدام روش، همیشه بهتر از بقیه عمل می‌کند.
 بنابراین در هر مساله، باید ابتدا
ناحیه مورد نظر را رسم کرده و سپس با توجه به آن، بهترین روش را انتخاب کنیم.

\section{مشتق‌های جزئی}
تابع $f$ را در $x_1$ مشتق‌پذیر گوییم، اگر $f'(x_1)$ وجود داشته باشد.
تابع $f$ را روی بازهٔ $I$ مشتق‌پذیر گوییم، اگر $f$ به ازای هر عدد واقع در این بازه، مشتق‌پذیر باشد.
اگر فرض کنیم  $f(x)=3x^2+12$ باشد، مشتق آن‌را حساب کنید. 
اگر $x$ عددی در دامنهٔ $f$ باشد، با استفاده از مطالب قبلی داریم

\begin{align*}
f'(x)&=\lim_{\Delta x\rightarrow 0}\dfrac{f(x+\Delta x) - f(x)}{\Delta x}\\[2mm]
&=\lim_{\Delta x\rightarrow 0}\dfrac{3(x+\Delta x)^{2}+12 - (3x^2+12)}{\Delta x}\\[2mm]
&=\lim_{\Delta x\rightarrow 0}\dfrac{3x^2+6x\Delta x+3(\Delta x)^{2}+12-3x^2-12}{\Delta x}\\[2mm]
&=\lim_{\Delta x\rightarrow 0}\dfrac{6x\Delta x+3(\Delta x)^{2}}{\Delta x}\\[2mm]
&=\lim_{\Delta x\rightarrow 0}6x+3(\Delta x)\\[2mm]
&=6x
\end{align*}
و لذا مشتق تابع $f$ به دست می‌آید.
\section{بسط تیلور}
ناحیه ممکن است  دارای شکل خاصی باشد که در این صورت، با استفاده از فرمول‌های  هندسه می‌توانیم مساحت\index{مساحت} آن‌را حساب کنیم.
حجم جسم حاصل از دوران ناحیهٔ بین محور $x$ها و  نمودار تابع پیوستهٔ $y=R(x)$،
$a\leq x\leq b$
 حول محور $x$ها برابر است با
 \begin{align}\label{a1eq1}
 V=\int_a^b \pi (R(x))^2 \di x
 \end{align}
\begin{example}
ناحیهٔ بین منحنی $y=\sqrt{x}$، 
$0\leq x\leq 4$
و محور $x$ها، برای تولید جسمی، حول محور $x$ها دوران داده می‌شود. حجم آن‌‌را
را پیدا کنید.
\end{example}
فرض کنید $y=f(x)$ یک تابع پیوسته\index{تابع پیوسته} روی بازهٔ $[a,b]$ باشد. این بازه را به $n$ زیربازه با انتخاب $n-1$ نقطه مانند $x_1$، $x_2$، $\ldots$، $x_{n-1}$ بین $a$ و $b$ تقسیم می‌کنیم. 
حجم جسم حاصل از دوران\index{جسم حاصل از دوران} ناحیهٔ بین محور $y$ها و  نمودار تابع پیوستهٔ\index{تابع پیوسته} $x=R(y)$،
$c\leq y\leq d$
 حول محور $y$ها برابر است با
 \begin{align}\label{a1eq2}
 V=\int_c^d \pi (R(y))^2 \di y
 \end{align}
 
\section{مختصات قطبی}
اگر در رابطهٔ \eqref{a1eq2} قرار دهیم $x_1+\Delta x=x$، آنگاه عبارت
 «$\Delta x\rightarrow 0$» 
معادل «$x\rightarrow x_1$»
است. بنابراین با توجه به فرمول \eqref{a1eq2} می‌توان نوشت
\begin{align}\label{a1deri3}
f'(x_1)=\lim_{ x\rightarrow x_1}\dfrac{f(x) - f(x_1)}{x-x_1}
\end{align}
مشتق تابع $f(x)=3x^2+12$ را در نقطهٔ $x=2$ حساب کنید. لذا مشتق تابع $f$ در نقطهٔ $x=2$ به دست می‌آید.

\begin{ptheorem1}
طبق قضیهٔ مقدار میانگین، می‌دانیم توابعِ با مشتق یکسان، در یک عدد ثابت
 با یکدیگر اختلاف دارند. به عبارت دیگر، اگر  دو تابع $f$ و $g$، مشتق یکسانی داشته باشند، آنگاه $f(x)=g(x)+C$ است. بنابراین می‌توان نتیجه 
 گرفت که هر گاه یک ضدمشتق $F$ برای تابع $f$ پیدا کنیم، ضدمشتق‌های دیگر $f$ در یک ثابت، با $F$ تفاوت دارند.
 می‌توان نتیجه 
 گرفت که هر گاه یک ضدمشتق $F$ برای تابع $f$ پیدا کنیم، ضدمشتق‌های دیگر $f$ در یک ثابت، با $F$ تفاوت دارند. این حالت را در انتگرال‌گیری با 
\[
 \int f(x)dx= F(x)+C
\]
 نشان می‌دهیم. وقتی ناحیه‌ای توسط منحنی‌هایی که یکدیگر را قطع می‌کنند، مشخص می‌شود، 
نقاط تقاطع\index{نقاط تقاطع}، حدود انتگرال‌گیری را تعیین می‌کنند. مثال بعدی، نمونه‌ای از این 
حالت را نشان می‌دهد. سوالی که ممکن است در اینجا پیش بیاید این است که کدام‌یک از ۳ روش گفته شده، بهتر است؟ واقعیت این است که به طور قطع، نمی‌توان گفت که کدام روش، همیشه بهتر از بقیه عمل می‌کند.
 بنابراین در هر مساله، باید ابتدا
ناحیه مورد نظر را رسم کرده و سپس با توجه به آن، بهترین روش را انتخاب کنیم.
\end{ptheorem1}
تابع $f$ را در $x_1$ مشتق‌پذیر گوییم، اگر $f'(x_1)$ وجود داشته باشد.
تابع $f$ را روی بازهٔ $I$ مشتق‌پذیر گوییم، اگر $f$ به ازای هر عدد واقع در این بازه، مشتق‌پذیر باشد.
اگر $f(x)=3x^2+12$ باشد، تعیین کنید که $f$ در کجا مشتق‌پذیر است؟
از آنجایی که $f'(x)=6x$  و $6x$ برای تمام اعداد حقیقی موجود است، لذا نتیجه
می‌شود که $f$ در همه جا مشتق‌پذیر است.
اگر تابع $f$ در $x_1$ تعریف شده باشد، آنگاه مشتق راست $f$ در $x_1$ با $f'_{+}(x_1)$ نشان داده
می‌شود و به صورت 
\begin{align}
f'_{+}(x_1)=\lim_{\Delta x\rightarrow 0^{+}}\dfrac{f(x_1+\Delta x) - f(x_1)}{\Delta x}
\end{align}
و یا به عبارت دیگر،
\begin{align}
f'_{+}(x_1)=\lim_{ x\rightarrow x_1^{+}}\dfrac{f(x) - f(x_1)}{x-x_1}
\end{align}
تعریف می‌شود؛ به شرطی که این حدود موجود باشند.
اگر تابع $f$ در $x_1$ تعریف شده باشد، آنگاه مشتق چپ $f$ در $x_1$ با $f'_{-}(x_1)$ نشان داده
می‌شود و به صورت 
\begin{align}
f'_{-}(x_1)=\lim_{\Delta x\rightarrow 0^{-}}\dfrac{f(x_1+\Delta x) - f(x_1)}{\Delta x}
\end{align}
و یا به عبارت دیگر،
\begin{align}
f'_{-}(x_1)=\lim_{ x\rightarrow x_1^{-}}\dfrac{f(x) - f(x_1)}{x-x_1}
\end{align}
تعریف می‌شود؛ به شرطی که این حدود موجود باشند.
\section{بردارها در فضا و خواص آن‌ها}
در این بخش چند حکم را با هم مرور می‌کنیم. 
مقدار  $\int_0 ^1\sqrt{1+\cos x} dx $ نمی‌تواند ۲ باشد.
به دلیل سادگی برهان، به عنوان تمرین به خواننده واگذار می‌شود.


اگر  $f(x)=3x^2+12$ باشد، مشتق آن‌را حساب کنید.
اگر $x_1$ عدد خاصی از دامنهٔ $f$ باشد، آنگاه می‌توان نوشت
\begin{align}\label{a1deri2}
f'(x_1)=\lim_{\Delta x\rightarrow 0}\dfrac{f(x_1+\Delta x) - f(x_1)}{\Delta x}
\end{align}
البته به شرطی که این حد وجود داشته باشد. از رابطهٔ \eqref{a1deri2} برای محاسبهٔ مشتق تابع $f$ در یک نقطهٔ 
خاص مانند $x_1$ استفاده می‌شود. 


فرض کنید تابع $f$ به صورت 
\[
f(x)=\left\{\begin{array}{ll}
2x-1& x<3\\
8-x & 3\leq x
\end{array} \right.
\]
تعریف شده است. پیوستگی و مشتق‌پذیری این تابع را در نقطهٔ $x=3$ بررسی کنید.
برای بررسی پیوستگی، سه شرط پیوستگی را بررسی می‌کنیم. (۱) داریم $f(3)=5$. بنابراین شرط اول که همان
موجود بودن $f(3)$ است، برقرار است. (۲) برای بررسی شرط دوم داریم
\[
\lim _{x\rightarrow 3^{-}}f(x)=\lim_{x\rightarrow 3^{-}} (2x-1)=5
\]
و 
\[
\lim _{x\rightarrow 3^{+}}f(x)=\lim_{x\rightarrow 3^{+}} (8-x)=5.
\]
بنابراین
 $\lim_{x\rightarrow 3}f(x)=5$
و لذا شرط دوم هم برقرار است. (۳)
$\lim_{x\rightarrow 3}f(x)=f(3)$.
بنابراین $f$ در $3$ پیوسته است. حال مشتق‌پذیری $f$ را در $3$ بررسی می‌کنیم. داریم
\begin{align*}
f'_{-}(3)&=\lim_{\Delta x\rightarrow 0^{-}}\dfrac{f(3+\Delta x)-f(3)}{\Delta x}\\
&=\lim_{\Delta x\rightarrow 0^{-}}\dfrac{(2(3+\Delta x)-1)-5}{\Delta x}\\
&=\lim_{\Delta x\rightarrow 0^{-}}\dfrac{6+2\Delta x-6}{\Delta x}\\
&=\lim_{\Delta x\rightarrow 0^{-}}\dfrac{2\Delta x}{\Delta x}\\
&=\lim_{\Delta x\rightarrow 0^{-}} 2\\
&=2.
\end{align*}
%\end{ntfullwidth}
\section{ضرب برداری}
تابع $f$ را در $x_1$ مشتق‌پذیر گوییم، اگر $f'(x_1)$ وجود داشته باشد.
تابع $f$ را روی بازهٔ $I$ مشتق‌پذیر گوییم، اگر $f$ به ازای هر عدد واقع در این بازه، مشتق‌پذیر باشد.هر گاه یک ضدمشتق $F$ برای تابع $f$ پیدا کنیم، ضدمشتق‌های دیگر $f$ در یک ثابت، با $F$ تفاوت دارند.
 می‌توان نتیجه 
 گرفت که هر گاه یک ضدمشتق $F$ برای تابع $f$ پیدا کنیم، ضدمشتق‌های دیگر $f$ در یک ثابت، با $F$ تفاوت دارند. این حالت را در انتگرال‌گیری با 
\[
 \int f(x)dx= F(x)+C
\]
 نشان می‌دهیم.
\begin{lstlisting}[caption={روش به دست آوردن انتگرال نامعین},label={codea1}]  
\newcommand{\@tufte@lof@line}[2]{%
  \leftskip 0.0em
  \rightskip 0em
  \parfillskip 0em plus 1fil
  \parindent 0.0em
  \@afterindenttrue
  \interlinepenalty\@M
  \leavevmode
  \@tempdima 2.0em
  \advance\leftskip\@tempdima
  \null\nobreak\hskip -\leftskip
  {#1}\nobreak\qquad\nobreak#2%
  \par%
}                                        
\end{lstlisting} 
