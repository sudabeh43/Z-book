\chapter{معرفی Z}\label{chapter2}
\paragraphfootnotes
Z یک زبان توصیف رسمی مدل گراست که در دهه 80 توسط گروه پژوهشی برنامه نویسی دانشگاه آکسفورد، توسعه داده شد.این زبان مبتنی بر تئوری مجموعه Zermelo است. Z در سال 2002، استاندارد ISO را دریافت کرد.از آن زمان تاکنون، Z در طیف گسترده ای از نرم افزارهای سیستمی مانند سیستم های پایگاه داده، سیستم های تراکنشی، سیستم های محاسبات توزیع شده و سیستم عامل ها بکار برده شده است. موفقیت قابل توجه Z در توصیف فاصل برنامه نویسی کاربردی CICS بود که بوسیله آزمایشگاه IBM در پارک Hursley انجام شد. تقریبا 3700 خط کد توسط Z تولید شد. این پروژه یک پروژه صنعتی بود که توصیف آن توسط Z منجر به کاهش 5.2 درصدی خطا نسبت به حالتی که توصیف Z وجود نداشته باشد، گردید. توصیف های Z ریاضیاتی هستند و از منطق دومقداری استفاده می کنند. استفاده از ریاضیات، صحت این زبان را تضمین می کند و به شناسایی تناقضات موجود در توصیف ها، کمک می کند.
Z یک رویکرد مدل گرا است که یک مدل صریح از حالت ماشین انتزاعی را نشان می دهد. عملگرها در این حالت تعریف شده اند. ریاضیات، در Z برای نشان گذاری توصیف های رسمی و حساب شِما، برای ساختار این توصیف ها بکار می روند. شِماها از نظر بصری قابل توجه هستند و بخش اصلی آنها شامل جعبه هایی است. شِماها برای توصیف حالات و عملیات بکار می روند. حساب شِما، به شِماها این قابلیت را می دهد که با دیگر شِماها ترکیب شوند و یا در کنار هم بلوک ها را بسازند.تصویر 1، یک شِمای ساده را توصیف می کند. این شِما، توصیف ریشه مرتبه دوم مثبت یک عدد حقیقی است. 

عملگرهای شِما بصورت پیش شرط/پس شرط تعریف می شوند. یک پیش شرط باید قبل از اینکه عملگر اجرا شود، بررسی شود در حالیکه پس شرط، پس از اجزای عملگر بررسی می شود. مسلما پس از اثبات درستی پیش شرط ها، عملگر اجرا خواهد شد.پیش شرط بصورت ضمنی در داخل عملگر تعریف شده است. هر عملگر یک فرض اثبات شده به همراه دارد که تضمین می کند درصورت درست بودن پیش شرط، عملگر، تغییرناپذیری سیستم را حفظ کند. تغییرناپذیری یکی از ویژگی های سیستم است که همواره و در هر زمانی باید درست باشد. حالت اولیه سیستم نیز بنحوی است که تغیرناپذیری سیستم برآورده شود.
\begin{figure}
\centering
\begin{schema}{SqRoot}
num? , root!: \mathbb{R}
\where
num? \geq 0\\
root! \expon 2 = num?\\
root! \geq 0
\end{schema}
\caption{توصیف تابع جذر}
\label{SqRoor}
\end{figure}
در شکل \ref{SqRoor}، پیش شرط توصیف تابع جذر بصورت 
$num? \ge 0$
است. بنابراین تابع $SqRoot$ تنها ریشه اعداد حقیقی مثبت را بدست می آورد. پس شرط های تابع جذر 
 $root!^2 = num?$ و $ root! \ge 0$ 
  هستند. این دو شرط بیان می کنند که اولا ریشه عددی مثبت است و ثانیا توان دو ریشه برابر با عدد ورودی است.
  
Z یک زبان مبتنی بر نوع است ، به این معنا که وقتی متغیری معرفی می شود، باید نوع آن نیز مشخص گردد. یک نوع، مجموعه ای از اشیا است. چندین نوع استاندارد در Z وجود دارند. این انواع عبارتند از اعداد طبیعی $\mathbb{N}$، اعداد صحیح $\mathbb{Z}$، و اعداد حقیقی $\mathbb{R}$. اعلان یک متغیر به نام x که از نوع X است، بصورت x:X انجام می گیرد.همچنین در Z امکان تعریف نوع توسط برنامه نویس نیز وجود دارد.

در توصیف های Z از قراردادهای مختلفی استفاده می شود، برای مثال $v?$ بیان کننده این است که v یک متغیر ورودی است و $v!$ بیانگر این است که v، یک متغیر خروجی است. در تابع $SqRoot$ که در بالا تعریف شد، $num?$ یک متغیر ورودی، و $root!$ یک متغیر خروجی را اعلان می کند. علامت $\Xi$ در یک شِما، نشاندهنده این است که عملگر، حالت را تغییر نمی دهد، درحالیکه علامت $\Delta$ بیانگر این است که عملگر، باعث تغییر حالت می گردد.

بسیاری از انواع داده های مورد استفاده در Z، مشابهی در زبان های برنامه نویسی استاندارد ندارند. بنابراین لازم است که ساختمان داده های توافقی، شناسایی و توصیف شوند تا درنهایت برای نمایش ساختارهای ریاضیاتی انتزاعی بکار روند. باتوجه به اینکه ساختارهای توافقی ممکن است با انتزاع تفاوت داشته باشند، عملگرهای مربوط به ساختار داده های انتزاعی، نیازمند پالایش به عملگرهای  داده های توافقی هستند. این پالایش باعث می شود که نتایج حاصل، یکسان گردد. برای سیستم های ساده، پالایش مستقیم امکان پذیر است. برای بیشتر سیستم های پیچیده، پالایش با تاخیر، بکار برده می شود که در آن یک دنباله از توصیف های توافقی افزاینده، برای توصیف های قابل اجرا، تولید می شوند.
\begin{figure}
\centering
\begin{schema}{Library}
on\_shelf, missing, borrowed : \mathbb{P} Bkd\_Id
\where
on\_shelf \cap missing = \oslash\\
on\_shelf \cap borrowed = \oslash \\
borrowed \cap missing = \oslash
\end{schema}
\caption{توصیف یک سیستم کتابخانه}
\label{Library}
\end{figure}

\begin{figure}
\centering
\begin{schema}{Borrow}
\bigtriangleup Library\\
b?:Bkd\_Id
\where
b? \in on\_shelf = on\_shelf \hide \{b?\}\\
borrowed' = borrowed \cup \{b?\}
\end{schema}
\caption{توصیف عملگر امانت گرفتن کتاب}
\label{Borrow}
\end{figure}
تصویر \ref{Borrow} نشاندهنده توصیف Z برای امانت گرفتن کتاب از یک سیستم کتابخانه است. کتابخانه شامل کتاب هایی است که در قفسه قرار دارند، کتاب هایی که به امانت رفته اند و کتاب هایی که گم شده اند. توصیف، با استفاده از مجموعه هایی که نشان دهنده کتاب های موجود در قفسه، به امانت رفته و گمشده است، کتابخانه را مدلسازی می کند. بنابراین سه زیرمجموعه مجزا از مجموعه کتاب ها وجود دارد. $Bkd-Id$ شناسه هر کدام را مشخص می کند.
\\
وضعیت سیستم با استفاده از شِمای $Library$ در شکل \ref{Library} مشخص شده است. دو عمل $Borrow$ و $Return$ می توانند بر روی حالت سیستم تاثیر بگذارند. عملگر$ Borrow$ در شکل 2-1 توصیف شده است.
نشانگذاری $\mathbb{P} Bkd-Id$ مجموعه توانی $Bkd-Id$ (مجموعه تمام زیرمجموعه های$ Bkd-Id$) را نشان می دهد. شرط مجزا بودن سه زیرمجموعه on-shelf، missing و borrowed در شِمای $Library$ تعریف شده است. این شرط با تهی بودن اشتراک دو به دوی این مجموعه ها مشخص شده است. 
پیش شرط عمل $Borrow$ (امانت دادن) این است که کتاب در قفسه موجود باشد. پس شرط آن این است که کتاب به مجموعه کتاب های به امانت رفته اضافه شود و از مجموعه کتاب های موجود در قفسه، حذف شود.  







