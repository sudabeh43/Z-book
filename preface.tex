\mychapter{پیش‌گفتار}
با توجه به کاربرد و اهمیت روزافزون ریاضیات عمومی در کمک به درک و توجیه پدیده‌های علمی و نیز نظر به اینکه کتاب‌های ریاضی‌ای که تاکنون به زبان فارسی در رابطه با موضوع ریاضیات عمومی ترجمه یا تالیف شده‌ است، نیازهای فعلی جامعه ریاضی و علمی را برآورده نمی‌کند، 
تصمیم  به تالیف کتاب حاضر  گرفته شد.


سطح این کتاب به گونه‌ای است که  برای دانشجویان سال اول دوره کارشناسی رشته ریاضی و دانشجویان کارشناسی رشته‌های فیزیک، مکانیک و سایر رشته‌های مرتبط قابل استفاده است.

 از ویژگی‌های این کتاب، توجه به سرفصل‌های درس نظریه ریاضیات همومی در دوره کارشناسی  است؛ به گونه‌ای  که تمامی سرفصل‌های مصوب وزارت علوم، تحقیقات و فناوری با بیانی ساده و قابل فهم آورده شده است. همچنین، با توجه به تعدد مثال‌ها، کتاب،  به صورت خودخوان نیز 
 قابل استفاده  است.

  کتاب حاضر از شش فصل تشکیل شده است. در فصل اول، مفاهیم و مقدمات اولیه مورد بررسی قرار گرفته و نیز قضیه اساسی وجودی و منحصر بفردی جواب بیان شده است.
 
 در فصل دوم، مباحث و مطالب فصل اول، روی سیستم معادلات دیفرانسیل مرتبه اول، توسیع داده شده است. همچنین در این فصل، سه روش مختلف برای حل سیستم معادلات ارایه شده است. لازم به ذکر است که روش حل سیستم معادلات با استفاده از روش جردن، بیشتر برای دوره‌های کارشناسی ارشد
 آورده شده است. لذا برای دوره‌های کارشناسی می‌توان از مطالعه این روش، چشم‌پوشی کرد. در ادامه فصل، معادلات دیفرانسیل مرتبه $n$ام و قضیه‌های مربوط به آن، بررسی شده‌ است.
 
 فصل سوم در ارتباط با مسایل مقدار مرزی و نظریه اشتورم است. در این فصل، قضیه‌های اساسی در ارتباط با مسایل مقدار مرزی، از جمله قضیه مقایسه‌ای  و قضیه تفکیک آورده شده است. 
 
 در فصل چهارم، سیستم‌های دینامیکی معرفی شده‌ است. تعاریف و مفاهیم نقاط ثابت، پایداری نقاط ثابت و تصویر فاز، با بیانی ساده و روان ارایه شده است. 
 
 فصل پنجم، درباره سیستم‌های دینامیکی خطی در صفحه بحث می‌کند. به بیان دقیق‌تر، سیستم‌های خطی متعارف و سیستم‌های خطی ساده
 در صفحه، بیان و تصاویر فاز مربوط به آن‌ها مورد کاوش قرار گرفته است.
 
 فصل ششم درباره سیستم‌های غیرخطی در صفحه است. در واقع  این فصل، دربرگیرنده مطالب تکمیلی فصل پنجم است. بیشتر مطالب این فصل، برای دوره‌های تحصیلات تکمیلی مناسب است.
 
 از ویژگی‌های این کتاب، توجه به سرفصل‌های درس نظریه ریاضیات همومی در دوره کارشناسی  است؛ به گونه‌ای  که تمامی سرفصل‌های مصوب وزارت علوم، تحقیقات و فناوری با بیانی ساده و قابل فهم آورده شده است. همچنین، با توجه به تعدد مثال‌ها، کتاب،  به صورت خودخوان نیز 
 قابل استفاده  است.

امید است که خوانندگان گرامی، نظرها و پیشنهاد‌های خود را با ما در میان گذاشته تا در چاپ‌های بعدی موجب غنی‌تر شدن کتاب گردد.


\vspace*{1\baselineskip}
\begin{flushleft}
وحید دامن‌افشان\\
کرمانشاه، تابستان ۹۸
\end{flushleft}