\mychapter{پاسخ تمرین‌های برگزیده}
\paragraphfootnotes

\solsection{۱}
\begin{psolutions}
\item[\ref{p1-1}]
$6x$
\item[\ref{p1-3}]
$2$
\item[\ref{p1-4}]
با مشتق‌گیری داریم
$f'(2)=6(2)=12$
\item[\ref{p1-5}]
تابع $f$ در همه جا مشتق‌پذیر است.
\item[\ref{p1-6}]
$-6$. نزول می‌کند.
\item[\ref{p1-7}]
ابتدا عبارت را ساده می‌کنیم:
\[y=(x-1)(x+1)=x^2 -1\]
بنابراین $y'=2x$.
\item[\ref{p1-10}]
$1/2$
\item[\ref{p1-13}]
بعد از ساده کردن عبارت، نتیجه می‌شود
$y'=3$.
\item[\ref{p1-14}]
با استفاده از مطالب گفته‌شده نتیجه می‌شود 
$f'(0)=-2$
\item[\ref{p1-15}]
$dy/dx=3xy-2y$
\item[\ref{p1-16}]
$x'(t)=12t^3+2t$ و $y'(t)=12t-1$.
\item[\ref{p1-20}]
$y=2x-6$
\item[\ref{p1-21}]
با استفاده از تعریف مشتق تابع وارون، می‌توان نوشت
$(f^{-1})'(11)=-3$
\item[\ref{p1-22}]
$2y^2=y''y-3y'^2$
\item[\ref{p1-23}]
$y^4+y'' y-3y'^2=0$.
\end{psolutions}

\solsection{۲}
\begin{psolutions}
\item[\ref{p2-1}]
$6x$
\item[\ref{p2-2}]
با استفاده از مطالب گفته‌شده و نیز تعریف مشتق نتیجه می‌شود 
$2$
\item[\ref{p2-3}]
با مشتق‌گیری داریم
$f'(2)=6(2)=12$
\item[\ref{p2-4}]
تابع $f$ در همه جا مشتق‌پذیر است.
\item[\ref{p2-5}]
$-6$. نزول می‌کند.
\end{psolutions}